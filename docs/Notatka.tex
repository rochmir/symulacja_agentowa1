\documentclass{article}
\usepackage{graphicx}
\usepackage[T1]{fontenc}
\usepackage[utf8]{inputenc}
\usepackage[polish]{babel}

\title{Notatka Projekt Obiektówka}


\begin{document}
\section*{Opis klas i metod projektu}

\subsection*{Klasa \texttt{Symulacja}}
Odpowiada za uruchomienie i sterowanie całą symulacją.
\begin{itemize}

  \item \texttt{main} -- pobiera dane od użytkownika (rozmiar planszy, liczba obiektów), tworzy planszę i uruchamia symulację.
  \item \texttt{uruchomSymulacje} -- wykonuje kolejne kroki symulacji: wyświetla planszę, zapisuje statystyki do pliku, kończy gdy nie ma kotów lub myszy.
  \item \texttt{wczytajLiczbe} -- pobiera liczbę od użytkownika, sprawdza czy jest w odpowiednim zakresie i czy jest liczbą całkowitą.
  \item \texttt{wyswietlKomunikatZakonczenia} -- wypisuje na końcu odpowiedni komunikat w zależności od tego, czy wyginęły koty, myszy, czy oba.
\end{itemize}

\subsection*{Klasa \texttt{Plansza}}
Przechowuje wszystkie obiekty na planszy: koty, myszy, norki, sery.
\begin{itemize}
  \item \texttt{symulujKrok} -- wykonuje jeden krok symulacji: dodaje nowe myszy z norek, porusza zwierzęta, sprawdza zjadanie, usuwa martwe zwierzęta.
  \item \texttt{wyswietlPlansze} -- rysuje planszę w konsoli, pokazuje liczbę każdego typu obiektu.
  \item \texttt{getKoty}, \texttt{getMyszy}, \texttt{getNorki}, \texttt{getSery} -- zwracają listy odpowiednich obiektów.
  \item \texttt{getLiczbaZywychMyszy} -- zwraca liczbę wszystkich żywych myszy (na planszy i tych, które jeszcze nie wyszły z norek).
\end{itemize}

\subsection*{Klasa \texttt{CSVWriter}}
Zajmuje się zapisem statystyk symulacji do pliku CSV.
\begin{itemize}
  \item \texttt{zapisz} -- dodaje nowy wiersz z danymi (krok, liczba kotów, liczba myszy).
  \item \texttt{zamknij} -- zamyka plik po zakończeniu zapisu.
\end{itemize}

\subsection*{Klasa \texttt{ObiektPlanszy}}
Klasa bazowa dla wszystkich obiektów na planszy, przechowuje współrzędne.
\begin{itemize}
  \item \texttt{getX}, \texttt{getY} -- zwracają aktualne współrzędne obiektu.
  \item \texttt{setX}, \texttt{setY} -- ustawiają nowe współrzędne.
  \item \texttt{wyswietl} -- wyświetla symbol obiektu na planszy.
\end{itemize}

\subsection*{Klasa \texttt{Norka}}
Reprezentuje norkę, z której może wyjść mysz.
\begin{itemize}
  \item \texttt{czyMyszWyszla} -- sprawdza, czy z tej norki już wyszła mysz.
  \item \texttt{ustawMyszWyszla} -- ustawia informację, czy mysz wyszła z norki.
  \item \texttt{wyswietl} -- wyświetla symbol norki.
\end{itemize}

\subsection*{Klasa \texttt{Ser}}
Reprezentuje ser na planszy, który mogą zjeść myszy lub koty.
\begin{itemize}
  \item \texttt{wyswietl} -- wyświetla symbol sera.
\end{itemize}

\subsection*{Klasa \texttt{Zwierze}}
\begin{itemize}
  \item Klasa bazowa dla zwierząt (kot, mysz), przechowuje energię.
  \item \texttt{poruszajSie} -- abstrakcyjna metoda ruchu zwierzęcia.
  \item \texttt{czyZywy} -- sprawdza, czy zwierzę ma jeszcze energię (czy żyje).
\end{itemize}

\subsection*{Klasa \texttt{Kot}}
Reprezentuje kota na planszy.
\begin{itemize}
  \item \texttt{poruszajSie} -- losowo przesuwa kota po planszy, zmniejsza energię.
  \item \texttt{zjedzMysz} -- kot odzyskuje maksymalną energię po zjedzeniu myszy.
  \item \texttt{czyZywy} -- sprawdza, czy kot ma energię.
  \item \texttt{wyswietl} -- wyświetla symbol kota.
\end{itemize}

\subsection*{Klasa \texttt{Mysz}}
Reprezentuje mysz na planszy.
\begin{itemize}
  \item \texttt{poruszajSie} -- losowo przesuwa mysz po planszy, zmniejsza energię, ustawia mysz jako aktywną.
  \item \texttt{czyAktywna} -- sprawdza, czy mysz wyszła już z norki.
  \item \texttt{ustawAktywna} -- ustawia aktywność myszy.
  \item \texttt{czyZywy} -- sprawdza, czy mysz ma energię.
  \item \texttt{zjedzSer} -- mysz odzyskuje maksymalną energię po zjedzeniu sera.
  \item \texttt{wyswietl} -- wyświetla symbol myszy.
\end{itemize}

\end{document}
